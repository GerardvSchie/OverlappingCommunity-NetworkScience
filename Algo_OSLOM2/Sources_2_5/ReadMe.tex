\documentclass[11pt]{article}
\usepackage{geometry}                % See geometry.pdf to learn the layout options. There are lots.
\geometry{letterpaper}                   % ... or a4paper or a5paper or ... 
%\geometry{landscape}                % Activate for for rotated page geometry
%\usepackage[parfill]{parskip}    % Activate to begin paragraphs with an empty line rather than an indent
\usepackage{graphicx}
\usepackage{amssymb}
\usepackage{epstopdf}
\usepackage[usenames,dvipsnames]{color}
\DeclareGraphicsRule{.tif}{png}{.png}{`convert #1 `dirname #1`/`basename #1 .tif`.png}

\title{User's guide for OSLOM (version 2.5)}
%\date{}                                           % Activate to display a given date or no date

\begin{document}
\maketitle
%\section{}
%\subsection{}

Thanks for downloading the code which implements OSLOM.


\section{Compiling}

The program comes along with a file called \textbf{compile\_all.sh}.

Type  

{ \textbf{./compile\_all.sh} }

from a Unix (MAC) terminal. If you are using Windows, you could still run
the program by installing MinGW
(Minimalist GNU for Windows, {\tt http://www.mingw.org/}).
If you see something like \textbf{./compile\_all.sh: Permission denied}, type:

\textbf{chmod 744 compile\_all.sh} 

which makes the script executable and try again with:

\textbf{./compile\_all.sh}. 

\section{What you get}

After compiling, you should see a number of binary
files. \textbf{oslom\_undir} and \textbf{oslom\_dir} 
implement OSLOM for undirected and directed networks. The option {\bf
  -f} introduces the name of the file where the network is stored. Try:

{ \textbf{./oslom\_undir -f example.dat -uw} }

and when the program is done (it takes about 15 seconds on a laptop computer), type: 

{ \textbf{./pajek\_write\_undir example.dat}}

Now look at the folder \textbf{example.dat\_oslo\_files}. You should
see a few files, described in the coming subsections.

\subsection{The modules}
\label{sec2.1}

The most important files are \textbf{tp} and \textbf{tp1} (\textbf{tp} is also copied in the main folder, for your convenience).

In \textbf{tp} you find the set of (overlapping) modules at the lowest hierarchical level. The file content looks like this:
\\

\#module 0 size: 18 bs: 5.69337$e-$101

7 10 13 17 18 22 28 31 33 37 38 41 44 45 47 50 51 56


{\footnotesize The line starting with \# says that the module with id 0 has 18
nodes 
and score of 5.69337$e-$101, which is an estimation of 
the probability of finding a module 
like this one in a random network (very very low in this case). To be specific, this is in fact an upper bound of the real significance. In any case, 
the program stops when the score is below the $p-$value (note that it can be much much lower). 
The second line gives the nodes in the module.}
\\

You should also see something like:

\#module 6 size: 29 bs: 0.756234


1 2 3 4 9 15 21 23 27 30 34 43 46 48 49 53 60 63 68 70 72 73 74 78 79 81 83 93 1000 

{\footnotesize The score is higher than the $p-$value because the module contains a node which is non-significant. Running the program with option 
\textbf{-singlet} will tell that node $1000$ has not been
  assigned to any module, i.e. it forms a module on its own. In other words, it is a homeless node. We recommend to use this option only if you are interested in detecting noisy nodes.}
\\

\textbf{tp1} is the analogous for the first hierarchical level. In our example, this is all, there are no other hierarchical levels for the network stored
in {\it example.dat}. In general, if there are further levels, the
program will produce other files: {\bf tp2}, {\bf tp3}, etc..
\\

\subsection{The pajek files}


Now open the file called \textbf{pajek\_file\_0.net}. The format of
this file is the usual pajek format and you can use {\bf Pajek}
({\tt http://vlado.fmf.uni-lj.si/pub/networks/pajek/}) and/or {\bf Gephi}
({\tt http://gephi.org/}) to visualize the network. 
Each module of the lowest level has a different color. Overlapping nodes are
black, homeless nodes are white. 
The only difference with \textbf{pajek\_file\_1.net} is that it has different colors to highlight higher level modules.

If the visualization looks too small, try:

{ \textbf{./pajek\_write\_undir example.dat} 2.5 }

You would get another pajek file which would produce a picture twice
and a half times bigger. Type the command again, 
to have different visualizations (colors and positions are chosen through a stochastic algorithm).
\\


\textbf{N. B.}: All the output files are stored in the directory called
\textbf{[network file]\_oslo\_files}. 
If the directory is not empty it will be cleared, so be careful if you want to save some previous output files.

\subsection{What else?}

This is not fundamental, so if you are impatient, skip this and go to Section \ref{basic_op}.
\\

 Here is a list of the other output files:

\begin{enumerate}
  \item \textbf{net1} contains the edge list of the network of
    communities found at the lowest hierarchical level. For instance,
  
0	10	5

  
  means that there are five links between node 0 and node 10. The
  labels 0 and 10
  are the module ids of file \textbf{tp}. 
Likewise, file  \textbf{net2} stores the network of communities found
at the second hierarchical level, whose labels are those used
for the modules in file \textbf{tp1}, etc.. In the case of weighted networks, you would see four numbers like these:
  
 0 10 15.76 5
 
which means that there are five links between node 0 and 10 and the sum of the weights of these links is 15.76.
     
  
  \item  \textbf{short\_tp1} ({\bf short\_tp2}) gives the modules of
    the community
    network \textbf{net1} ({\bf net2}, {\bf net3}, ...) using its
    labels, which refer to the ids of the modules listed in
    \textbf{tp} ({\bf tp1}, {\bf tp2}, ...).
     
  
  \item \textbf{partitions\_level\_0(1)}  contains the pruned modules the program found at the end of each run. 
  
  \item \textbf{statistics\_level\_0(1)}  contains some basic statistics of the modules found for each hierarchical level.

  
  \item \textbf{pos\_0} is written by \textbf{pajek\_write\_undir}. It
    gives the position of the nodes of \textbf{example.dat}, the
    format is \textbf{x y node}. \textbf{pos\_1} is the analogous for the
    nodes in \textbf{net1}, \textbf{pos\_2} for the nodes in
    \textbf{net2}, etc..
  
  
\end{enumerate}


\section{OSLOM's Options}


  \subsection{Basic Options}
\label{basic_op}


\begin{enumerate}
  \item \textbf{Directed vs Undirected}. First of all, you need to choose between 
  
  \textbf{./oslom\_undir} 
  
  and  
  
  \textbf{./oslom\_dir}. 
  
  As you can easily guess, the former is for the undirected networks,
  the latter for directed ones. 
The same holds for \textbf{pajek\_write\_undir} and  \textbf{pajek\_write\_dir}.
If you have a directed network, the format of the input file is
\textbf{node1 node2}, meaning that there is a link from 
\textbf{node1} to  \textbf{node2}.  Self loops and repetitions are ignored in both cases.

  \item \textbf{-uw (-w)} is the only mandatory flag. 
  
  Option \textbf{-uw} is for unweighted networks, or better it is for
  weighted networks where considering weights as multiple links. 
This means that it is possible to use a format for the network file like: 
  
  \textbf{node1 node2} 
  
  or 
  
  \textbf{node1 node2 $m_{12}$}
  
  
  but $m_{12}$ must be an integer number! It tells how many links are
  between the two nodes. In the null model, 
these links are randomly rewired along with all the links of the network.
  
  Option \textbf{-w} is to choose the weighted version of the
  algorithm. 
This selects a different null model where the strength of a node is shared between its neighbors. The input file can be: 
  
   \textbf{node1 node2 $w_{12}$} 
  
  or 
  
  \textbf{node1 node2 $w_{12}$ $m_{12}$}

  where $w_{12}$ is the weight between \textbf{node1} and
  \textbf{node2}. If there are $m_{12}$ multiple links between the two
  nodes, $w_{12}$ must be the sum of the weights on those links.  
  
  About the labels of the nodes, they must be non-negative integer
  numbers, but they do not have to be consecutive and can start from any number.
  
  

\end{enumerate}


  \subsection{ {Other Options}}
  
   
  \begin{enumerate}
  
 \item \textbf{-r $m$} sets equal to $m$ the number of runs for the first
      hierarchical level. The default value is 10. 
This value depends on your patience. The bigger $m$, the more accurate the
results. We are working to let the program choose the optimal number
of runs according to the convergence of the results.
The option {\bf -r 0} has to be employed when OSLOM starts from a
partition found by other algorithms.  
  
   \item \textbf{-hr $m$} sets the number of runs for higher
     hierarchical level equal to $m$. The default value is 50 (the method
     should be faster since the aggregated network is usually much
     smaller). If you are not interested in hierarchies set
     \textbf{-hr 0} and the method will stop after finding the lowest
     level partition.
  
\item \textbf{-seed $m$}. The program is not deterministic, so it
  needs a seed for its random number generator. 
Using this option it will be set equal to $m$, otherwise the seed 
will be set reading the file \textbf{time\_seed.dat}, which will be updated afterwards. 


 \item \textbf{-hint filename}: reads a file with a partition
    previously found by another algorithm. The file has to be
    formatted like \textbf{tp}  (lines starting with \# are skipped,
    so they are optional): it simply has to contain the modules
    separated by an end of line. All the modules read from the file
    will be cleaned up by the algorithm, 
which would retain only the significant ones. Then, the program would
look for submodules and cleanup all of them. 
The outcoming clusters will be written in file
\textbf{partitions\_level\_0}, together with the 
modules found during the cluster search (unless you set \textbf{-r 0}). This option can be used only for the first hierarchical level.



\item \textbf{-load filename}: reads a file with a partition
previously found by OSLOM. The file must have the format of the output
file \textbf{tp} (see Section~\ref{sec2.1}). 
In particular, this option can be useful if you want to parallelize
the program. You have to run 
OSLOM in different folders with different seeds (option \textbf{-seed}).  
  



\item \textbf{-t $l$} sets the $p-$value equal to $l$. The default
  value is $0.10$. Interestingly, if you increase this value you get
  more modules. The reason is that submodules are more easily considered significant.

\item \textbf{-singlet}: singletons. The program usually finds a
  number of nodes which are not assigned to any module. By default, 
  the  program  assigns ``homeless" nodes to a module
  according to the score of the node
  with respect to the existing modules:
the module with best score will ``accept'' the node (this only applies to the lowest hierarchical level).  
Setting this flag, instead, homeless nodes are not assigned to any cluster.
Also, a file called \textbf{tp\_without\_singletons} will be written with all the nodes assigned (as by default). 
This option should be used if you are interested in filtering nosily nodes (typically low-degree nodes).
         
     \item \textbf{-cp $l$} sets the \textit{coverage parameter}  equal to $l$. This parameter is a kind of resolution parameter: it is used to decide between taking some modules or their union. Default value is 0.5. Bigger value leads to bigger clusters. $l$ must be in the interval $(0, 1)$.
  
      \item \textbf{-fast}: this flag is to have fast results. It is equivalent to set  \textbf{-r $1$ -hr $1$}, so it enables the fastest possible execution of the program.
   
   
     \item \textbf{-infomap $r$}: the program will call another program
     (\textbf{infomap}) and will apply OSLOM's cleanup procedure
     to the modules found by it. $r$ is the number of times \textbf{infomap} will be called, good values are between 1 and 10.
    
    
   \item \textbf{-copra $r$}, \textbf{-louvain $r$} are similar flags
     for other programs. They can be used simultaneously. The general
     idea is that the more programs are used, the better the
     exploration of possible modules will be, but of course more time is needed.
   
   
   
   
   
   \end{enumerate}
   
   
   
  
  
  Examples:
  \\
  
{ \textbf{./oslom\_undir  -f example.dat -uw -infomap 3 -copra 2 -louvain 1 -r 2 } }
  
This will run oslom on example.dat using our method to explore the graph (two runs) and the outputs of infomap after 3 iterations, of copra after 2 and of louvain as initial conditions  [Oslom cleans-up afterwards the found modules]. The final output is formed by the best modules found with all these methods.   

  { \textbf{./oslom\_undir -f example.dat -uw -infomap 1 -r 0 -hr 0} }
  
OSLOM runs on example.dat using as initial modules the output of infomap. Since $\textrm{r} = 0$ and $\textrm{hr}=0$, our module search technique is not applied.
\\
  
  



  
  
   
    \textbf{N. B.}: All the external programs have been distributed
because they are free. However, 
if you use one or more of them for your research, please cite the
paper where the method is described.

    
  
  


\section{References}


If you used this program for your research, please cite this paper: (to appear)

\vskip0.3cm

\noindent In addition, please cite the following papers if you used  \textbf{-copra}, \textbf{-louvain} or \textbf{-infomap}.

\begin{enumerate}

  \item if you used \textbf{-copra}, please cite: Steve Gregory,
    New J. Phys. {\bf 12}, 103018 (2010).
  \item if you used \textbf{-infomap}, please cite: M. Rosvall and
    C. T. Bergstrom, Proc. Natl. Acad. Sci. U.S.A {\bf 105}, 1118 (2008).
\item if you used \textbf{-louvain}, please cite:   V. D. Blondel,
  J.-L. Guillaume, R. Lambiotte, and E. Lefebvre, J. Stat. Mech. {\bf
    P10008} (2008).
\end{enumerate}

Further information and the source codes can be found in the following urls:

\begin{enumerate}

  \item \textbf{copra}: {\tt http://www.cs.bris.ac.uk/$\sim$steve/networks/software/copra.html}
  \item \textbf{infomap}: {\tt http://www.tp.umu.se/$\sim$rosvall/code.html}
  \item \textbf{louvain}: {\tt http://sites.google.com/site/findcommunities/}
  \end{enumerate}






















\end{document}  
         
